
  
\documentclass{exam}
\usepackage[utf8]{inputenc}
\usepackage{upgreek}
\usepackage[margin=1in]{geometry}
\usepackage{amsmath,amssymb}
\usepackage{multicol}
\usepackage{stmaryrd}
\usepackage{graphicx}
\usepackage{caption}
\usepackage{tikz}
\usepackage{dsfont}
\usepackage{enumitem}
\usepackage{hyperref}
\usepackage{float}
\usetikzlibrary{matrix}
\newcommand\tab[1][1cm]{\hspace*{#1}}
\pagestyle{head}
\firstpageheader{}{}{}
\runningheader{\examnum}{\class}{\name}
\runningheadrule
\newcommand{\class}{Fundamentos de bases de datos}
\newcommand{\term}{Facultad de Ciencias UNAM}
\newcommand{\examnum}{Practica 04}
\newcommand{\examdate}{04/04/2022}
\newcommand{\name}{Jurassic Team}
\begin{document}

\noindent
\begin{tabular*}{\textwidth}{l @{\extracolsep{\fill}} r @{\extracolsep{6pt}} l}
\textbf{\class} & \textbf{\term}\\
\textbf{\examnum} & \textbf{\name}\\
\textbf{\examdate}
\end{tabular*}\\
\rule[2ex]{\textwidth}{2pt}

\section*{Entidades y atributos}

\begin{itemize}
	\item \textit{Estetica}: Es una entidad fuerte. Tiene como llave \underline{IdEstetica}. Sus atributos simples son: \texttt{Nombre, ServicioObservacion, ServicioBaño, Telefono, NumConsultorios} y \texttt{Horario}. Tiene un solo atributo compuesto: \texttt{Direccion} con componentes: \texttt{Estado, Calle, Numero} y \texttt{CP}. Tiene un atributo calculado \texttt{Ingresos}.
	\item \textit{Trabajador}: Es una entidad fuerte. Tiene como llave \underline{CURP}. Sus atributos simples son: \texttt{Telefono, Salario, FechaNac} y \texttt{Genero}. Sus atributos compuestos son: \texttt{Direccion} con componentes: \texttt{Estado, Calle, Numero, CP} y \texttt{NombreC} con componentes: \texttt{Paterno, Materno, Nombre}. Tiene un atributo calculado: \texttt{Edad}.
	\item \textit{Supervisor} Es una especialización de \textit{Trabajador}. Sus atributos simples son: \texttt{Periodo, RFC, HoraFin} y \texttt{HoraInicio}. Tiene un atributo calculado: \texttt{HorasTrabajo}.
	\item \textit{Veterinario} Es una especialización de \textit{Trabajador}. Sus atributos simples son: \texttt{PacientesAct, RFC, HoraFin} y \texttt{HoraInicio}. Tiene un atributo calculado: \texttt{HorasTrabajo}.
	\item \textit{Estilista} Es una especialización de \textit{Trabajador}. Tiene un atributo simple: \texttt{Certificado}.
	\item \textit{Producto}: Es una entidad fuerte. Tiene como llave \underline{Codigo}. Sus atributos simples son: \texttt{Nombre, Precio, Caducidad, ArchivoImg, Tipo, Descripcion} y \texttt{cantidadInventario}.
	\item \textit{Recibo}: Es una entidad fuerte. Tiene como llave \underline{NumRecibo}.
	\item \textit{ReciboProducto}: Es una especialización de \textit{Recibo}.
	\item \textit{ReciboServicio}: Es una especialización de \textit{Recibo}.
	\item \textit{Dueño} es una entidad fuerte. Tiene como llave: \underline{CURP}. Sus atributos simples son: \texttt{FechaNac, EsFrecuente, Telefono} y \texttt{Email}. Sus atributos compuestos son: \texttt{Direccion} con componentes: \texttt{Estado, Calle, Numero, CP} y \texttt{NombreC} con componentes: \texttt{Paterno, Materno, Nombre}.
	\item \textit{Tarjeta} es una entidad débil respecto a \textit{Dueño}. Su discriminante es \underline{NumTarjeta}. Tiene como atributos simples: \texttt{Titular} y \texttt{FechaVencimiento}.
	\item \textit{Mascota} es una entidad débil respecto a \textit{Dueño}. Su discriminante es \underline{Nombre}. Tiene como atributos simples: \texttt{Edad, Peso Raza} y \texttt{Especie}.
	\item \textit{Servicio}: Es una entidad fuerte. Tiene como llave \underline{IdServicio}. Tiene un solo atributo simple: \texttt{Precio}.
	\item \textit{TratEstetico}: Es una especialización de \textit{Servicio}. Tiene un atributo simple \texttt{Tipo}.
	\item \textit{Consulta}: Es una especialización de \textit{Servicio}.
	\item \textit{Emergencia}: Es una especialización de \textit{Consulta}. Tiene como atributos simples: \texttt{Procedimiento, Codigo} y \texttt{Sintomas}
	\item \textit{Normal}: Es una especialización de \textit{Consulta}. Tiene como atributos simples: \texttt{FechaRevision, Motivo} y \texttt{Estado}. Tiene un atributo multivaluado: \texttt{Medicamentos}.
\end{itemize}

\section*{Restricciones del modelo}
    
\begin{itemize}
	\item \textbf{Supervisar} tiene restricción de participación total y es de cardinalidad, uno con ambas entidades: \textit{Estetica} y \textit{Supervisor}.
	\item \textbf{Trabajar} tiene restricción de participación total y es de cardinalidad muchos a muchos con ambas entidades: \textit{Estetica} y \textit{Trabajador}.
	\item \textbf{Vender} tiene restricción de participación parcial y es de cardinalidad muchos a muchos con ambas entidades: \textit{Estetica} y \textit{Producto}.
	\item \textbf{Registrar} tiene restricción de participación total del lado de \textit{Recibo} y participación parcial del lado de \textit{Estética}. Con cardinalidad uno del lado de \textit{Estetica} y muchos del lado de \textit{Recibo}
	\item \textbf{Anexar} tiene restricción de participación total del lado de \textit{ReciboProducto} y participación parcial del lado de \textit{Producto}. Con cardinalidad muchos en ambos lados
	\item \textbf{Asociar} tiene restricción de participación total del lado de \textit{Tarjeta} y participación parcial del lado de \textit{Dueño}. Con cardinalidad uno del lado de \textit{Dueño} y muchos del lado de \textit{Tarjeta}
	\item \textbf{Pagar} tiene restricción de participación total de ambos lados \textit{Recibo} y \textit{Estética}. Con cardinalidad uno del lado de \textit{Dueño} y muchos del lado de \textit{Recibo}
	\item \textbf{Pertenecer} tiene restricción de participación total del lado de \textit{Mascota} y participación parcial del lado de \textit{Dueño}. Con cardinalidad uno del lado de \textit{Dueño} y muchos del lado de \textit{Mascota}
	\item \textbf{Realizar} tiene restricción de participación total del lado de \textit{Servicio} y participación total del lado de \textit{Mascota}. Con cardinalidad uno del lado de \textit{Mascota} y muchos del lado de \textit{Servicio}
	\item \textbf{Listar} tiene restricción de participación total de ambos lados \textit{Recibo} y \textit{ReciboServicio}. Con cardinalidad uno del lado de \textit{ReciboServicio} y muchos del lado de \textit{Servicio}
	\item \textbf{Procurar} tiene restricción de participación total del lado de \textit{TratEstetico} y participación parcial del lado de \textit{Estilista}. Con cardinalidad uno del lado de \textit{Estilista} y muchos del lado de \textit{TratEstetico}
	\item \textbf{Atender} tiene restricción de participación total del lado de \textit{Consulta} y participación parcial del lado de \textit{Veterinario}. Con cardinalidad uno del lado de \textit{Veterinario} y muchos del lado de \textit{Consulta}
\end{itemize}

\section*{Consideraciones de diseño}

\begin{itemize}
	\item La relación \textbf{Supervisar} entre \textit{Estetica} y \textit{Supervisor} es de tipo total y uno a uno en ambos sentidos debido a que se asume que hay un único supervisor por \textit{Estetica} y que este solo supervisa una sola \textit{Estetica}.
	\item Se consideran dos tipos de \textit{Recibo} para diferenciar entre compras de productos y consumo de algún servicio.
	\item La entidad \textit{Trabajador} tiene como relación \textbf{Trabajar} con la entidad \textit{Estetica} de manera muchos a muchos para permitir que haya transferencias de trabajadores en caso de ser necesario. Por ejemplo un veterinario haciendo una consulta especial en otra estética porque un paciente se mudó o llego de emergencia.
	\item Siguiendo la misma logica del punto anterior, \textit{Trabajador} no se considera entidad debil ya que permite registrar a una persona previo a considerar una \textit{Estetica} donde va a \textbf{Trabajar}.
	\item La entidad de \textit{Tarjeta} se considera débil debido a que no se debe permitir registrar ninguna tarjeta que no pertenezca a un \textit{Dueño}.
	\item Los atributos \texttt{ServicioBaño} y \texttt{ServicioObservacion} representan valores booleanos que indican si el respectivo tipo de servicio se ofrece en la \textit{Estetica}. Solo se consideran estos dos servicios debido a que son los estipulados en el documento de requerimientos pero existe posibilidad de agregar atributos adicionales si existen otros servicios.
\end{itemize}

\end{document}
